\section{Hypergraph-Embeddable Optimization Problems}\label{sec:hypergraph-embeddableOptimizationProblems}

\subsection{Least Squares Optimization Problems}\label{subsec:leastSquaresOptimizationProblems}

\subsubsection{Definition}
A least squares optimization problem with a set of constraints $\mathcal{C}$ can be generally formulated as:
\begin{align}
    \label{formulation}
    F(\chi) &= \sum_{k\in\mathcal{C}}e_k(\chi_k, z_k)^{T}\Omega_k e_k(\chi_k, z_k)\\
    {\chi}^* &= \text{argmin}_{\chi} F(\chi),
\end{align}
where
\begin{align}
    \chi = (\chi_1^T,\dots,\chi_n^T)^T
\end{align}
is the structured vector of parameters subject to optimization and
\begin{align}
    \chi_k = (\chi_{k_1}^T,\dots,\chi_{k_q}^T)^T
\end{align}
is the structured vector of parameter blocks subject to the constraint $k\in\mathcal{C}$.

$z_k$ and $\Omega_k$ denote the mean and information matrix of constraint $k$, respectively.

Finally, $e_k(\chi_k, z_k)$ is a vector error function measuring how well the parameter blocks
$\chi_k$ satisfy the constraint $k$.

\subsubsection{Linearization of the Error Function}
Given an initial guess $\hat{\chi}$ of the parameter set $\chi$, we can expand the error function
by means of a first order \textsc{Taylor} expansion:
\begin{align}
    e_k(\chi_k) &= e_k(\hat{\chi_k}+\Delta\chi_k)\\
    &\approx e_k(\hat{\chi_k})+J_{k}\Delta\chi_k,
\end{align}
where $J_{k}=\nabla e_k(\hat{\chi_k})$ is the \textsc{Jacobian} computed at $\hat{\chi}_k$.

Insertion into\ \ref{formulation} yields:
\begin{align}
    F_k(\hat{\chi_k}+\Delta\chi_k) &= e_k(\hat{\chi_k}+\Delta\chi_k)^{T}\Omega_k e_k(\hat{\chi_k}+\Delta\chi_k)\\
    &\approx \left(e_k(\hat{\chi_k})+J_{k}\Delta\chi_{k}\right)^T\Omega_k\left(e_k(\hat{\chi_k})+J_{k}\Delta\chi_{k}\right)\\
    &= \underbrace{e_k(\hat{\chi_k})^T\Omega_k e_k(\hat{\chi_k})}_{c_k}
    + 2\underbrace{e_k(\hat{\chi_k})^{T}\Omega_{k}J_k}_{b_k}\Delta \chi_k
    + \Delta\chi_k^T \underbrace{J_k^T\Omega_k J_k}_{H_k}\Delta \chi_k\\
    &= c_k + 2b_{k}\Delta \chi_k + \Delta \chi_k^T H_k \Delta \chi_k.
\end{align}
Generalizing this result to the set of all constraints yields:
\begin{align}
    F(\hat{\chi}+\Delta \chi) &= \sum_{k\in\mathcal{C}} F_k(\hat{\chi_k}+ \Delta\chi_k)\\
    &= \sum_{k\in\mathcal{C}} c_k + 2b_{k}\Delta \chi_k + \Delta \chi_k^T H_k \Delta \chi_k\\
    &= c + 2b^T\Delta\chi + \Delta \chi^T H \Delta \chi.
    \label{target}
\end{align}
where
\begin{align}
    c &= \sum_{k\in\mathcal{C}}c_{k}\\
    b &= \bigoplus_{k\in\mathcal{C}}b_{k}\\
    \Delta \chi &= \bigoplus_{k\in\mathcal{C}}\Delta \chi_{k}\\
    H &= \bigoplus_{k\in\mathcal{C}} H_{k}.
\end{align}
Here, $\bigoplus$ denotes an addition of blocks for vectors and matrices, where the position is derived from the constraint index.

We need to minimize $F(\hat\chi+\Delta\chi)$ in $\Delta\chi$. It can be done by solving the linear system
\begin{align}
    H\Delta\chi^\star = -b
\end{align}

\subsubsection{Example}
Let us consider a system of two parameter blocks $\chi_0$, $\chi_1$ (e.g.\ poses in two dimensions)
and three constraints $z_0$, $z_1$ (e.g.\ position measurements), $z_{01}$ (e.g.\ an odometry measurement), where the first two constraints
control only one parameter block each and the
last constraint connects the two parameter blocks.

We will set $z_{0}=(0,1)^T, z_1 = (1,0)^T, z_{01} = (0.5, -0.5)^T$.
Thus, we have a system, where the position measurements are further apart
than the odometry requires.

The information matrices are defined as follows:
\begin{align}
    \Omega_{0} &= \begin{pmatrix}
                      10 & 0 \\ 0 & 10
    \end{pmatrix}\\
    \Omega_{1} &= \begin{pmatrix}
                      10 & 0 \\ 0 & 10
    \end{pmatrix}\\
    \Omega_{01} &=\begin{pmatrix}
                      1 & 0 \\ 0 & 1
    \end{pmatrix}
\end{align}
That is, the odometry measurement is considered ten times less precise than the position measurement.

We will define the error functions
\begin{align}
    e_0(\chi_0,z_0) &= z_0 - \chi_0 \\
    e_1(\chi_1,z_1) &= z_1 - \chi_1 \\
    e_{01}(\chi_{01},z_{01}) &= z_{01} - (\chi_1 - \chi_0).
\end{align}
Thus, for the \textsc{Jacobian}s we find
\begin{align}
    J_0 &= \begin{pmatrix}
               \frac{\partial (z_0 - \chi_0)_x}{\partial (\chi_0)_x} & \frac{\partial (z_0 - \chi_0)_x}{\partial (\chi_0)_y} \\
               \frac{\partial (z_0 - \chi_0)_y}{\partial (\chi_0)_x} & \frac{\partial (z_0 - \chi_0)_y}{\partial (\chi_0)_y}
    \end{pmatrix}\\
    &= \begin{pmatrix}
           1 & 0 \\ 0 & 1
    \end{pmatrix}\\
    J_1 &= \begin{pmatrix}
               1 & 0 \\ 0 & 1
    \end{pmatrix}\\
    J_{01} &= \begin{pmatrix}
                  \frac{\partial (z_0 - (\chi_1 - \chi_0))_x}{\partial (\chi_0)_x} &
                  \frac{\partial (z_0 - (\chi_1 - \chi_0))_x}{\partial (\chi_0)_y} &
                  \frac{\partial (z_0 - (\chi_1 - \chi_0))_x}{\partial (\chi_1)_x} &
                  \frac{\partial (z_0 - (\chi_1 - \chi_0))_x}{\partial (\chi_1)_y} \\
                  \frac{\partial (z_0 - (\chi_1 - \chi_0))_y}{\partial (\chi_0)_x} &
                  \frac{\partial (z_0 - (\chi_1 - \chi_0))_y}{\partial (\chi_0)_y} &
                  \frac{\partial (z_0 - (\chi_1 - \chi_0))_y}{\partial (\chi_1)_x} &
                  \frac{\partial (z_0 - (\chi_1 - \chi_0))_y}{\partial (\chi_1)_y}
    \end{pmatrix}\\
    &= \begin{pmatrix}
           -1 & 0 & 1 & 0 \\ 0 & -1 & 0 & 1
    \end{pmatrix}
\end{align}
Let us apply the position constraints as initial guesses of the pose variables:
\begin{align}
    \hat{\chi_0} &= z_0 = (0,1)^{T}\\
    \hat{\chi_1} &= z_1 = (1,0)^{T}.
\end{align}
Now we can calculate the components of our target equation\ \ref{target}:
\begin{align}
    b_0 &= e_0^T \cdotp \Omega_0 J_{0}\\
    &= (0,0) \cdotp \begin{pmatrix}
                        10 & 0 \\ 0 & 10
    \end{pmatrix} \cdotp \begin{pmatrix}
                             1 & 0 \\ 0 & 1
    \end{pmatrix}\\
    &= (0,0).\\
    b_1 &= e_1^T \cdotp \Omega_{1} J_{1}\\
    &= (0,0) \cdotp \begin{pmatrix}
                        10 & 0 \\ 0 & 10
    \end{pmatrix} \cdotp \begin{pmatrix}
                             1 & 0 \\ 0 & 1
    \end{pmatrix}\\
    &= (0,0).\\
    b_{01} &= e_{01}^T \cdotp \Omega_{01} J_{01}\\
    &= \left( (0.5, -0.5) - \left( (1,0) - (0,1) \right) \right)\cdotp \begin{pmatrix}
                                                                           1 & 0 \\ 0 & 1
    \end{pmatrix} \cdotp \begin{pmatrix}
                             1 & 0 & -1 & 0 \\ 0 & 1 & 0 & -1
    \end{pmatrix}\\
    &= (-0.5, 0.5)\cdotp \begin{pmatrix}
                             1 & 0 \\ 0 & 1
    \end{pmatrix} \cdotp \begin{pmatrix}
                             1 & 0 & -1 & 0 \\ 0 & 1 & 0 & -1
    \end{pmatrix}\\
    &= (-0.5, 0.5, 0.5, -0.5).\\
    \Rightarrow b &= b_0 \oplus b_1 \oplus b_{01}\\
    &= (0,0) \oplus (0,0) \oplus (-0.5, 0.5, 0.5, -0.5)\\
    &= (-0.5, 0.5, 0.5, -0.5)\\
    H_0 &= J_{0}^T \Omega_0 J_{0}\\
    &= \begin{pmatrix}
           -1 & 0 \\ 0 & -1
    \end{pmatrix} \cdotp \begin{pmatrix}
                             10 & 0 \\ 0 & 10
    \end{pmatrix} \cdotp \begin{pmatrix}
                             -1 & 0 \\ 0 & -1
    \end{pmatrix}\\
    &= \begin{pmatrix}
           10 & 0 \\ 0 & 10
    \end{pmatrix}.\\
    H_1 &= J_{1}^T \Omega_1 J_{1}\\
    &= \begin{pmatrix}
           10 & 0 \\ 0 & 10
    \end{pmatrix}.\\
    H_{01} &= J_{01}^T \Omega_{01} J_{01}\\
    &= \begin{pmatrix}
           1 & 0 \\ 0 & 1 \\ -1 & 0 \\ 0 & -1
    \end{pmatrix} \cdotp \begin{pmatrix}
                             1 & 0 \\ 0 & 1
    \end{pmatrix} \cdotp \begin{pmatrix}
                             1 & 0 & -1 & 0 \\ 0 & 1 & 0 & -1
    \end{pmatrix}\\
    &= \begin{pmatrix}
           1 & 0 & -1 & 0\\ 0 & 1 & 0 & -1 \\ -1 & 0 & 1 & 0 \\ 0 & -1 & 0 & 1
    \end{pmatrix}.\\
    \Rightarrow H &= H_0 \oplus H_1 \oplus H_{01}\\
    &= \begin{pmatrix}
           10 & 0 \\ 0 & 10
    \end{pmatrix} \oplus \begin{pmatrix}
                             10 & 0 \\ 0 & 10
    \end{pmatrix} \oplus \begin{pmatrix}
                             1 & 0 & -1 & 0\\ 0 & 1 & 0 & -1 \\ -1 & 0 & 1 & 0 \\ 0 & -1 & 0 & 1
    \end{pmatrix}\\
    & = \begin{pmatrix}
            11 & 0 & -1 & 0\\ 0 & 11 & 0 & -1 \\ -1 & 0 & 11 & 0 \\ 0 & -1 & 0 & 11
    \end{pmatrix}
\end{align}
Now, the last step is to solve
\begin{align}
    H\Delta \chi^* &= -b\\
    \begin{pmatrix}
        11 & 0 & -1 & 0\\ 0 & 11 & 0 & -1 \\ -1 & 0 & 11 & 0 \\ 0 & -1 & 0 & 11
    \end{pmatrix} \Delta \chi^* &=
    \begin{pmatrix}
        0.5 \\ -0.5 \\ -0.5 \\ 0.5
    \end{pmatrix} \\
    \Rightarrow \Delta \chi^* &= \frac{1}{24} \begin{pmatrix}
                                                  1 \\ -1 \\ -1 \\ 1
    \end{pmatrix}.
\end{align}
and apply the update
\begin{align}
    \chi^{(1)} &= \hat{\chi} + \Delta \chi^{*}\\
    &= (0,1,1,0)^T + \frac{1}{24}(1, -1, -1, 1)^{T}\\
    &\approx (0.04, 0.96, 0.96, 0.04).
\end{align}
Thus, our pose variables moved a little closer together, driven by the odometry constraint.

