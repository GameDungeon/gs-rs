\section{Optimization Algorithm}
In the following sections, the optimization algorithm will be introduced. Firstly, the structure's main iteration steps will be introduced. Most of the work happens in the first step, the calculation of $H$ and $b$. Subsequently, the calculations which apply to all dimensions and factor types will be presented.



\subsection{Iteration Steps}
\label{iter_steps}
Given a specific number of iterations $n$ and the initial guess $x_i^{(0)}$ for each variable, the optimizer algorithm will repeat the following steps $n$ times:
\begin{enumerate}
	\item Calculate $H$ and $b$ by setting them to $\boldsymbol{0}$, then looping through all factors and updating their non-fixed variables' entries in $H$ and $b$.
	\item Calculate $\Delta x$, the vector containing data about how much each current variable guess $x_i^{(k)}$ should be updated in this step, by solving the linear system
		\begin{align}
			H \Delta x = -b^T.
		\end{align}
	\item Update the guesses for each non-fixed variable $x_i$ with
		\begin{align}
			x_i^{(k+1)} = x_i^{(k)} + \Delta x_i.
		\end{align}
		In the case of 2D variables with a rotation, normalize the rotation to $[-\pi, \pi)$.
		In the case of 3D variables with a rotation, normalize the rotation of $\Delta x_i$. As explained in section~\ref{optimization_3d}, the quaternion data will only include the imaginary part. Therefore, set the scalar part to $1$ before normalizing. The $+$~operator then resembles the concatenation of two isometries, analogously as the $*$~operator in section~\ref{optimization_3d}.
\end{enumerate}



\subsection{Calculation of $\boldsymbol{H}$ and $\boldsymbol{b}$}
How exactly the variables' entries in $H$ and $b$ are updated depends on the factor type. In all cases the factor's increments on parts of $H$ and $b$, $H^{fac}$ and $b^{fac}$ respectively, will be computed as follows:
\begin{align}
H^{fac} = J^T * \Omega * J,
\end{align}
\begin{align}
b^{fac} = e^T * \Omega * J,
\end{align}
where $\Omega$, $J$ and $e$ are the factor's information matrix, Jacobian matrix and error vector, respectively. While $\Omega$ is a given constant of the factor, $J$ and $e$ have to be calculated for each factor in each iteration.

If the factor only involves the variable $x_i$, $H$ and $b$ are updated as follows:
\begin{align}
\label{H_inc}
H_{ii} = H_{ii} + H^{fac},
\end{align}
\begin{align}
\label{b_inc}
b_i = b_i + b^{fac},
\end{align}
where the subscripts of $H$ and $b$ denote the row and column index of the submatrix or subvector assigned to the respective variable. If the factor involves two variables $x_i$ and $x_j$, $H^{fac}$ and $b^{fac}$ will have the structure
\begin{align}
H^{fac} =
\begin{pmatrix}
\boldsymbol{H}^{fac}_{ii} & \boldsymbol{H}^{fac}_{ij}\\
\boldsymbol{H}^{fac}_{ji} & \boldsymbol{H}^{fac}_{jj}
\end{pmatrix}
\end{align}
and
\begin{align}
b^{fac} =
\begin{pmatrix}
\boldsymbol{b}^{fac}_{i} & \boldsymbol{b}^{fac}_{j}
\end{pmatrix},
\end{align}
respectively, such that $H_{mn}$ will be incremented by $H^{fac}_{mn}$ and $b_n$ will be incremented by $b^{fac}_n$, analogously to equations (\ref{H_inc}) and (\ref{b_inc}). Fixed variables are excluded from $H$ and $b$ and therefore do not have any submatrices or subvectors which would need to be updated. In this case, these parts of $H^{fac}$ and $b^{fac}$ are simply ignored.

In the following sections, the calculation is described for all 2D and 3D factors supported by \textbf{gs-rs}.