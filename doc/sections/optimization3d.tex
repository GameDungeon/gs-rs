\section{Optimization in 3D}
\label{optimization_3d}
In 3D, rotations are often represented as quaternions. To reduce these to a minimal representation, only the imaginary part of the unit quaternions is saved within the solution $\Delta x$, while the scalar part is implicitly $1$~\cite{grisetti2010tutorial}. This is achieved by always using unit quaternions and removing the entry of the scalar part in the error vector $e$ after its computation.

In the following sections, the individual 3D factors' calculations of $J$ and $e$ are presented. Similarly as in section~\ref{optimization_2d}, the functions $pos(x)$ and $rot_Q(x)$ will be used to refer to the 3D position vector and the rotation quaternion of a 3D pose or isometry $x$. When referring to the rotation as a 3D rotation matrix, the function $rot_{RM}(x)$ will be used instead. When using $x$ in calculations, it will refer to the pose's 3D isometry containing both the position vector as translation as well as the rotation quaternion. The binary operator $*$ will be used to concatenate these isometries. The inverse of an isometry will be denoted as $x^{-1}$.

% TODO similarities in all 3D factors (skew, ...(?))
Some functions will be used to calculate gradients of a 3D isometry $x$, such as $\frac{dq}{dR}(x)$, which is computed as follows using the trace function $tr(M)$ of 3x3 matrices:
\begin{align}
	s = \frac{1}{2}\sqrt{tr(rot_{RM}(x)) + 1}
\end{align}
\begin{align}
	a_1 = -\frac{rot_{RM}(x)_{21}-rot_{RM}(x)_{12}}{32s^3}
\end{align}
\begin{align}
	a_2 = -\frac{rot_{RM}(x)_{02}-rot_{RM}(x)_{20}}{32s^3}
\end{align}
\begin{align}
	a_3 = -\frac{rot_{RM}(x)_{10}-rot_{RM}(x)_{01}}{32s^3}
\end{align}
\begin{align}
	b = \frac{1}{4s}
\end{align}
\begin{align}
	\frac{dq}{dR}(x) =
	\begin{pmatrix}
		a_1 & 0 &  0 &  0 & a_1 & b & 0 & -b & a_1\\
		a_2 & 0 & -b &  0 & a_2 & 0 & b &  0 & a_2\\
		a_3 & b &  0 & -b & a_3 & 0 & 0 &  0 & a_3
	\end{pmatrix}
\end{align}
It will always be assumed that the condition $tr(rot_{RM}(x)) > 0$ holds.

Another function used in this context is $skew(x)$, computed as follows using a 3D vector or position $x$:
\begin{align}
	skew(x) =
	\begin{pmatrix}
		   0 &  x_z & -x_y\\
		-x_z &    0 &  x_x\\
		 x_y & -x_x &    0
	\end{pmatrix},
\end{align}
where $x_x$, $x_y$ and $x_z$ are the first, second and third vector components, respectively.



\subsection{Position3D}
% TODO document optimization
The \textit{Position3D} factor involves one \textit{VehicleVariable} $x_v$. Given the measurement $x_m$, the Jacobian matrix $J$ is calculated as follows:
\begin{align}
	E = x_m^{-1} * x_v
\end{align}
\begin{align}
	\frac{dte}{dqj} = \frac{dq}{dR}(E) *
	\begin{pmatrix}
		\boldsymbol{skew(rot_{RM}(E)_0)}\\
		\boldsymbol{skew(rot_{RM}(E)_1)}\\
		\boldsymbol{skew(rot_{RM}(E)_2)}
	\end{pmatrix}
\end{align}
\begin{align}
	J =
	\begin{pmatrix}
		\boldsymbol{E} &               \boldsymbol{0}\\
		\boldsymbol{0} & \boldsymbol{\frac{dte}{dqj}}
	\end{pmatrix},
\end{align}
where $M_i$ refers to the column at index $i$ of the matrix $M$.

The error vector $e$ is computed as follows:
\begin{align}
	e = x_m^{-1} * x_v
\end{align}
As mentioned in section~\ref{optimization_3d}, the scalar part of the quaternion is removed in $e$. $e$ i then interpreted as a six-dimensional vector with the top three entries for its position and the bottom three entries for its rotation.



\subsection{Odometry3D}
The \textit{Odometry3D} factor involves two \textit{VehicleVariables} $x_i$ and $x_j$. Given the measurement $x_{ij}$, the Jacobian matrix $J$ is calculated as follows:
\begin{align}
	A = x_{ij}^{-1}
\end{align}
\begin{align}
	B = x_i^{-1} * x_j
\end{align}
\begin{align}
	E = A * B
\end{align}
\begin{align}
	\frac{dte}{dqi} = rot_{RM}(A) * skew(pos(B))
\end{align}
\begin{align}
	\frac{dre}{dqi} = \frac{dq}{dR}(E) *
	\begin{pmatrix}
		\boldsymbol{rot_{RM}(A) * skew(rot_{RM}(B)_0)^T}\\
		\boldsymbol{rot_{RM}(A) * skew(rot_{RM}(B)_1)^T}\\
		\boldsymbol{rot_{RM}(A) * skew(rot_{RM}(B)_2)^T}
	\end{pmatrix}
\end{align}
\begin{align}
	J_i =
	\begin{pmatrix}
	\boldsymbol{-rot_{RM}(A)} & \boldsymbol{\frac{dte}{dqi}}\\
	\boldsymbol{0} & \boldsymbol{\frac{dre}{dqi}}
	\end{pmatrix}
\end{align}
\begin{align}
	\frac{dre}{dqj} = \frac{dq}{dR}(E) *
	\begin{pmatrix}
		\boldsymbol{skew(rot_{RM}(E)_0)}\\
		\boldsymbol{skew(rot_{RM}(E)_1)}\\
		\boldsymbol{skew(rot_{RM}(E)_2)}
	\end{pmatrix}
\end{align}
\begin{align}
	J_j =
	\begin{pmatrix}
		\boldsymbol{rot_{RM}(E)} &               \boldsymbol{0}\\
		          \boldsymbol{0} & \boldsymbol{\frac{dre}{dqj}}
	\end{pmatrix}
\end{align}
\begin{align}
	J =
	\begin{pmatrix}
		\boldsymbol{J_i} & \boldsymbol{J_j}
	\end{pmatrix},
\end{align}
where $M_i$ refers to the column at index $i$ of the matrix $M$.

The error vector $e$ is computed as follows:
\begin{align}
	e = x_{ij}^{-1} * x_i^{-1} * x_j
\end{align}
As mentioned in section~\ref{optimization_3d}, the scalar part of the quaternion is removed in $e$. $e$ i then interpreted as a six-dimensional vector with the top three entries for its position and the bottom three entries for its rotation.



\subsection{Observation3D}
The \textit{Observation3D} factor involves one \textit{VehicleVariable} $x_i$ and one \textit{LandmarkVariable} $x_j$. In the following, the binary operator $*$ denotes the transformation of the position given by the right operand by the isometry given by the left operand. With the measurement $x_{ij}$, the Jacobian matrix $J$ is calculated as follows:
\begin{align}
	J =
	\begin{pmatrix}
		\boldsymbol{-I_3} & \boldsymbol{skew(pos(x_i^{-1} * pos(x_j)))^T} & \boldsymbol{rot_{RM}(x_i^{-1})}
	\end{pmatrix}
\end{align}
The error vector $e$ is computed as follows:
\begin{align}
	e = pos(x_i^{-1} * pos(x_j)) - pos(x_{ij})
\end{align}